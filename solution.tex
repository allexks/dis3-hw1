\documentclass{article}

\usepackage{amsfonts}
\usepackage{amsmath}
\usepackage[utf8]{inputenc}
\usepackage[bulgarian]{babel}

\title{Курсова работа №1}
\author{Александър Игнатов \\ Ф№ 62136 }
\date{\today}


\begin{document}

\maketitle

\section{Условие}

Дадена е рекурентната редица \( \{a_n\}_{n=1}^\infty \),
където за всяко \( n \in \mathbb{N} \),
\( a_{n+1}=\frac{2a_n^2 + a_n + 6}{a_n + 6} \) и \( a_1 = \lambda \in \mathbb{R} \backslash \{-6\} \).
Изследвайте за сходимост редицата \( \{a_n\}_{n=1}^\infty \) в зависимост от \(\lambda\).

\section{Решение}

Нека допуснем, че редицата е сходяща и има граница

\[
    l = \lim_{n \to \infty} a_n
\]

Чрез използване на граничен преход получаваме:

\begin{gather*}
    l = \frac{2l^2 + l + 6}{l + 6} \Longleftrightarrow (l-2)(l-3) = 0 \\
    l_1 = 2 \cup l_2 = 3
\end{gather*}

Пресмятаме:

\begin{gather}
    a_{n+1} - a_n = \frac{(a_n - 2)(a_n - 3)}{a_n + 6}
\end{gather}

Наблюдаваме знакът на израза \( (1) \):

\begin{align*}
    a_n &\in (-\infty, -6) &\Longrightarrow a_{n+1} - a_n < 0 \\
    a_n &\in (-6, 2) &\Longrightarrow a_{n+1} - a_n > 0 \\
    a_n &= 2 &\Longrightarrow a_{n+1} - a_n = 0 \\
    a_n &\in (2, 3) &\Longrightarrow a_{n+1} - a_n < 0 \\
    a_n &= 3 &\Longrightarrow a_{n+1} - a_n = 0 \\
    a_n &\in (2, +\infty) &\Longrightarrow a_{n+1} - a_n > 0
\end{align*}

Пресмятаме:

\begin{gather}
    a_{n+1} - (-6) = \frac{8a_n^2 + 7a_n + 42}{a_n + 6} \\
    a_{n+1} - 2 = \frac{2(a_n + \frac{3}{2})(a_n - 2)}{a_n + 6} \\
    a_{n+1} - 3 = \frac{2(a_n + 2)(a_n - 2)}{a_n + 6} \\
    2a_{n+1} - (-3) = \frac{4a_n^2 + 5a_n + 30}{a_n + 6} \\
    a_{n+1} - (-2) = \frac{2a_n^2 + 3a_n + 18}{a_n + 6}
\end{gather}

От уравнение (2):

\begin{gather*}
    8a_n^2 + 7a_n + 42 > 0,\, \forall a_n \Longrightarrow sign(a_{n+1} + 6) = sign(a_n + 6) \\
    \{a_n\}\downarrow ,\, a_n \in (-\infty, -6) \\
    \{a_n\}\uparrow ,\, a_n \in (-6, +\infty)
\end{gather*}

От уравнения (3) и (5):

\begin{gather*}
    \{a_n\}\downarrow ,\, a_n \in (-\infty, -6) \\
    \{a_n\}\uparrow ,\, a_n \in \left(-6, -\frac{3}{2}\right) \\
    a_n = 2,\, \forall n \in \mathbb{N} \backslash \{1\},\, a_1 = -\frac{3}{2} \\
    \{a_n\}\downarrow ,\, a_n \in \left(-\frac{3}{2}, 2\right) \\
    a_n = 2,\, \forall n \in \mathbb{N},\, a_1 = 2 \\
    \{a_n\}\uparrow ,\, a_n \in (2, +\infty)
\end{gather*}

От уравнения (4) и (6):

\begin{gather*}
    \{a_n\}\downarrow ,\, a_n \in (-\infty, -6) \\
    \{a_n\}\uparrow ,\, a_n \in (-6, -2) \\
    a_n = 3,\, \forall n \in \mathbb{N} \backslash \{1\},\, a_1 = -2 \\
    \{a_n\}\downarrow ,\, a_n \in (-2, 3) \\
    a_n = 3,\, \forall n \in \mathbb{N},\, a_1 = 3 \\
    \{a_n\}\uparrow ,\, a_n \in (3, +\infty)
\end{gather*}

Така получваме следното поведение за различните интервали на параметъра:

\begin{align*}
    \lambda &\in (-\infty, -6) &\Longrightarrow \{a_n\}\downarrow ,&\, \lim_{n \to \infty} a_n = -\infty \\
    \lambda &\in (-6, -2) &\Longrightarrow \{a_n\}\uparrow ,&\, \lim_{n \to \infty} a_n = +\infty \\
    \lambda &= -2 &\Longrightarrow a_n = 3,\, \forall n \in \mathbb{N}\backslash\{1\},&\, \lim_{n \to \infty} a_n = 3 \\
    \lambda &\in \left(-2, -\frac{3}{2}\right) &\Longrightarrow \{a_n\}\downarrow ,&\, \lim_{n \to \infty} a_n = 2 \\
    \lambda &= -\frac{3}{2} &\Longrightarrow a_n = 2,\, \forall n \in \mathbb{N}\backslash\{1\},&\, \lim_{n \to \infty} a_n = 2 \\
    \lambda &\in \left(-\frac{3}{2}, 2\right) &\Longrightarrow \{a_n\}\uparrow ,&\, \lim_{n \to \infty} a_n = 2 \\
    \lambda &= 2 &\Longrightarrow a_n = 2,\, \forall n \in \mathbb{N},&\, \lim_{n \to \infty} a_n = 2 \\
    \lambda &\in (2, 3) &\Longrightarrow \{a_n\}\downarrow ,&\, \lim_{n \to \infty} a_n = 2 \\
    \lambda &\in (3, +\infty) &\Longrightarrow \{a_n\}\uparrow ,&\, \lim_{n \to \infty} a_n = +\infty \\
\end{align*}

\section{Отговор}

\begin{equation*}
    \lim_{n \to \infty} a_n =
    \begin{cases}
        -\infty &,\, \lambda \in (-\infty, -6) \\
        2       &,\, \lambda \in (-2, 3) \\
        3       &,\, \lambda \in \{-2, 3\} \\
        +\infty &,\, \lambda \in (-6, -2) \cup (3, +\infty)
    \end{cases}
\end{equation*}

\end{document}