\documentclass{article}

\usepackage{amsfonts}
\usepackage{amsmath}
\usepackage[utf8]{inputenc}
\usepackage[bulgarian]{babel}

\title{Курсова работа №1}
\author{Александър Игнатов \\ Ф№ 62136 }
\date{\today}


\begin{document}

\maketitle

\section{Условие}

Дадена е рекурентната редица \( \{a_n\}_{n=1}^\infty \),
където за всяко \( n \in \mathbb{N} \),
\( a_{n+1}=\frac{2a_n^2 + a_n + 6}{a_n + 6} \) и \( a_1 = \lambda \in \mathbb{R} \backslash \{-6\} \).
Изследвайте за сходимост редицата \( \{a_n\}_{n=1}^\infty \) в зависимост от \(\lambda\).

\section{Решение}

Нека допуснем, че редицата е сходяща и има граница

\[
    l := \lim_{n \to \infty} a_n
\]

Чрез използване на граничен преход получаваме:

\begin{gather*}
    l = \frac{2l^2 + l + 6}{l + 6} \Longleftrightarrow (l-2)(l-3) = 0 \\
    l_1 = 2 \cup l_2 = 3
\end{gather*}

Пресмятаме:

\begin{gather}
    a_{n+1} - a_n = \frac{(a_n - 2)(a_n - 3)}{a_n + 6} \\
    a_{n+1} - 2 = \frac{2(a_n + \frac{3}{2})(a_n - 2)}{a_n + 6} \\
    a_{n+1} - 3 = \frac{2(a_n + 2)(a_n - 2)}{a_n + 6} \\
    2a_{n+1} + 3 = \frac{4a_n^2 + 5a_n + 30}{a_n + 6} \\
    a_{n+1} + 2 = \frac{2a_n^2 + 3a_n + 18}{a_n + 6}
\end{gather}

% TBD

\section{Отговор}

\begin{align*}
\lambda \in (-\infty, -6) &\Longrightarrow \lim_{n \to \infty} a_n = -\infty \\
\lambda \in (-2, 3) &\Longrightarrow \lim_{n \to \infty} a_n = 2 \\
\lambda \in \{-2, 3\} &\Longrightarrow \lim_{n \to \infty} a_n = 3 \\
\lambda \in (-6, -2) \cup (3, +\infty) &\Longrightarrow \lim_{n \to \infty} a_n = +\infty \\
\end{align*}

\end{document}